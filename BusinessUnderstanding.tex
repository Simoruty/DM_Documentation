\section{Business Understanding}
Questa fase si focalizza sulla individuazione degli obiettivi e i requisiti del progetto dal punto di vista del business.

\subsection{Background}
Ogni giorno, vengono inviate circa 25 milioni di email indesiderate, chiamate anche email di spam. Tale cifra corrisponde a quasi il 10 \% di tutte le email inviate nel mondo; inoltre, indagini svolte sull'argomento, rivelano che generalmente il 40 \% delle email ricevute giornalmente dai dipendenti di molte imprese risultano essere email di spam, arrivando in alcuni casi, anche al 90 \%.
Queste percentuali risultano essere molto elevate principalmente a causa della facile diffusione delle proprie caselle di posta verso qualunque tipo di contatto, diventando cosi bersagli di messaggi promozionali di qualunque tipo. Questa inondazione di messaggi di spam quindi genera due problemi principali:
\begin{itemize}
\item Saturazione della propria casella di posta, anche se al giorno d'oggi si dispone di una capienza elevata;
\item Perdita di tempo abbastanza considerevole da parte del ricevente nel filtrare queste email.
\end{itemize}


Nel corso degli anni, il desiderio di automatizzare il rilevamento e relativa selezione di queste email di spam, ha portato alla creazione e diffusione di numerosi progetti software e di prodotti commerciali in grado di filtrare lo spam in maniera tutto sommato efficiente; uno dei più diffusi è \textit{SpamAssassin} \footnote{\url{http://spamassassin.apache.org/}}, programma opensource rilasciato sotto licenza Apache 2.0. Si basa su regole di confronto del contesto, supporta anche regole basate su DNS, checksum e filtraggio statistico, inoltre supporta programmi esterni e database online.
SpamAssassin è considerato uno dei filtri antispam più efficaci, specialmente se usato congiuntamente con un database antispam.\cite{wiki:SpamAssassin}

\subsubsection{Risorse}
La principale risorsa utilizzata è l'hardware del sistema utilizzato per eseguire l'algoritmo di data mining, in particolar modo, un sistema Windows con un Quad-Core Intel i7 2.00 GHz e 4 GB di Ram.
Il tool di data mining scelto è WEKA (\textbf{W}aikato \textbf{E}nvironment for \textbf{K}nowledge \textbf{A}nalysis) \cite{WEKA}:
una popolare suite di software per il machine learning scritta in Java e sviluppata nell'Università di Waikato (Nuova Zelanda); è stato deciso di utilizzare tale suite, in quanto il software porta con sè i seguenti vantaggi :
\begin{itemize}
	\item Liberamente scaricabile dal sito \footnote{Weka site: \url{http://www.cs.waikato.ac.nz/ml/weka/downloading.html}};
    \item Portabile, in quanto totalmente implementato in java;
    \item Ampia gamma di tecniche di preprocessing e modellazione dei dati;
    \item Facile da usare grazie alla GUI;
\end{itemize}

Per quanto riguarda invece il personale umano, l'unica risorsa umana interpellata nella sperimentazione è lo sperimentatore stesso.

\subsubsection{Vincoli}
	Non sono presenti nè vincoli temporali, nè problemi legali legati alla diffusione del dataset in questione.

\subsubsection{Assunzioni}
	Si assume che i dati di cui si intende disporre siano liberamente accessibili e che non siano falsi o errati.

\subsection{Obiettivi di Business}
	Secondo quanto detto in precedenza, l'obiettivo di business di questo progetto consiste nell'individuazione delle email di spam attraverso l'utilizzo di tecniche di Data Mining.

\subsubsection{Task di Data Mining}
	Il task da realizzare è di tipo \textit{predittivo}, in particolar modo, sarà un task di \textit{classificazione}. L'obiettivo sarà quindi quello di creare un classificatore che sia in grado di etichettare correttamente le nuove email come \textit{spam} o \textit{nospam} sulla base del training set dato in pasto al classificatore. 
\subsection{Criteri di successo}

\subsection{Glossario dei termini}

\subsection{Analisi Costi-Benefici}

\subsection{Piano di Progetto}
\chapter{Data Understanding}
\section{Raccolta dei dati}
Il dataset preso in esame essendo oggetto della DMC 2003 Competition, è liberamente scaricabile dal sito \url{http://www.data-mining-cup.de/en/review/dmc-2003/}.


\section{Descrizione dei dati}
Il dataset è composto da 8000 istanze rappresentanti le email. Ogni istanza è caratterizzata da 834 attributi, di cui uno \textit{target}, atto ad etichettare l'istanza come \textit{spam} o \textit{no-spam}, ed un attributo identificativo numerico (\textit{id}). Esempi di attributi utilizzati a rappresentare le istanze delle email sono le seguenti:
%Training set: 8000 Istanze 834 Attributi = 833 + Target
%data\_dmc2003\_train.txt ... 

%Test set: 11177 Istanze da classificare 833 Attributi
%data\_dmc2003\_train.txt.
\begin{table}[hbtp]
	\begin{tabular}{ c | c | c}
		\textbf{Nome attributo} & \textbf{Tipo} & \textbf{Sottotipo} \\
		\hline
		id & Categorico & Nominale \\ 
		ACCEPT\_CREDIT\_CARDS & Categorico & Nominale \\ 
		ACCOUNT\_CLICK & Categorico & Nominale \\ 
		ACT\_NOW & Categorico & Nominale \\ 
		ADDRESSES\_ON\_CD & Categorico & Nominale \\ 
		ADULT\_SITE & Categorico & Nominale \\ 
		ADVERT\_CODE & Categorico & Nominale \\ 
		ADVERT\_CODE2 & Categorico & Nominale \\ 
		ALL\_CAPS\_HEADER & Categorico & Nominale \\ 
		ALL\_CAP\_PORN & Categorico & Nominale \\ 
		ALL\_NATURAL & Categorico & Nominale \\ 
		ALTA\_BUSCADORES\_ES & Categorico & Nominale \\ 
		AMATEUR\_PORN & Categorico & Nominale \\ 
		AMAZING & Categorico & Nominale \\ 
		AMAZING\_STUFF & Categorico & Nominale \\ 
		ANOTHER\_NET\_AD & Categorico & Nominale \\ 
		AOL\_USERS\_LINK & Categorico & Nominale \\ 
		APPLY\_ONLINE & Categorico & Nominale \\ 
		APPROVED\_BY & Categorico & Nominale \\ 
		\vdots  &  \vdots  &  \vdots  \\
	\end{tabular}
\end{table}

Tutti gli attributi, ad esclusione dell'\textit{id} e del \textit{target}, sono di tipo categorico nominale booleano, esso assumerà valore 0 qualora l'email non dovesse averlo, 1 il viceversa.
Il dataset, inoltre, è distribuito in un formato testuale.


\section{Verifica della qualità dei dati}

Bewertung der Ergebnisse
------------------------

Der Jury ist bekannt, welche von den 11.177 zu klassifizierenden
E-Mails tatsächlich Nicht-Spam-Mail oder Spam-Mail ist. Genauer
gesagt, stammen alle Daten aus einer Stichprobe von insgesamt
19.177 E-Mails.

Die eingesandten Ergebnisse werden mit der bekannten Information über
die tatsächliche Zuordnung der E-Mails verglichen und der Anteil der
nicht ausgefilterten Spam-Mails bestimmt. Gleichzeitig wird der Anteil
der ausgefilterten Nicht-Spam-Mails ermittelt und die Einhaltung
der 1 \% Klausel (siehe oben) überprüft. Sieger ist der Teilnehmer
oder die Teilnehmerin, welche(r) unter Einhaltung der 1 \% Klausel die
wenigsten Spam-Mails zustellt. Teilnehmer, die die 1 \% Klausel ver-
letzen, werden nicht gewertet.

\section{Esportazione dei dati}


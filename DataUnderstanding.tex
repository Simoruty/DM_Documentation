\chapter{Data Understanding}

\begin{figure}[hbtp]
	\centering
	\includegraphics[width=0.5\textwidth]{./images/CRISPDM_2.png}
	\caption{CRISP-DM - Data Understanding}
	\label{CRISPDM_2}
\end{figure}

\section{Raccolta dei dati}
Il dataset preso in esame essendo oggetto della DMC 2003 Competition, è liberamente scaricabile dal sito \url{http://www.data-mining-cup.de/en/review/dmc-2003/}. 

\section{Descrizione dei dati}
Il dataset è composto da 8000 istanze rappresentanti le email. Ogni istanza è caratterizzata da 834 attributi, di cui uno \textit{target}, atto ad etichettare l'istanza come \textit{spam} o \textit{no-spam}, ed un attributo identificativo numerico (\textit{id}). Esempi di attributi utilizzati a rappresentare le istanze delle email sono le seguenti:
%Training set: 8000 Istanze 834 Attributi = 833 + Target
%data\_dmc2003\_train.txt ... 

%Test set: 11177 Istanze da classificare 833 Attributi
%data\_dmc2003\_train.txt.
\begin{table}[hbtp]
	\begin{tabular}{ c | c | c}
		\textbf{Nome attributo} & \textbf{Tipo} & \textbf{Sottotipo} \\
		\hline
		id & Categorico & Nominale \\ 
		ACCEPT\_CREDIT\_CARDS & Categorico & Nominale \\ 
		ACCOUNT\_CLICK & Categorico & Nominale \\ 
		ACT\_NOW & Categorico & Nominale \\ 
		ADDRESSES\_ON\_CD & Categorico & Nominale \\ 
		ADULT\_SITE & Categorico & Nominale \\ 
		ADVERT\_CODE & Categorico & Nominale \\ 
		ADVERT\_CODE2 & Categorico & Nominale \\ 
		ALL\_CAPS\_HEADER & Categorico & Nominale \\ 
		ALL\_CAP\_PORN & Categorico & Nominale \\ 
		ALL\_NATURAL & Categorico & Nominale \\ 
		ALTA\_BUSCADORES\_ES & Categorico & Nominale \\ 
		AMATEUR\_PORN & Categorico & Nominale \\ 
		AMAZING & Categorico & Nominale \\ 
		AMAZING\_STUFF & Categorico & Nominale \\ 
		ANOTHER\_NET\_AD & Categorico & Nominale \\ 
		AOL\_USERS\_LINK & Categorico & Nominale \\ 
		APPLY\_ONLINE & Categorico & Nominale \\ 
		APPROVED\_BY & Categorico & Nominale \\ 
		\vdots  &  \vdots  &  \vdots  \\
	\end{tabular}
\end{table}

Tutti gli attributi, ad esclusione dell'\textit{id} e del \textit{target}, sono di tipo categorico nominale booleano, esso assumerà valore 0 qualora l'email non dovesse averlo, 1 il viceversa.
Il dataset, inoltre, è distribuito in un formato testuale.


\section{Verifica della qualità dei dati}

Il dataset contiene dati di buona qualità in quanto garantiscono le seguenti qualità:

\begin{itemize}
	\item \textbf{Accuratezza}:
	\item \textbf{Consistenza}:
	\item \textbf{Attualità dei dati}:
\end{itemize}

Accuratezza: I valori contenuti nei tre database non sono cambiati nel tempo a parte per quanto riguarda il database WEBKB che potrebbe avere subito un aggiornamento nel tempo, trattandosi di pagine web. Ad ogni modo ciò non rappresenta un problema visto che l’obiettivo del KDD process è quello di valutare due versioni di un algoritmo di Data Mining.
Completezza: Non sempre tutti i dati sono disponibili. Ci sono alcuni missing values nei tre database.

Consistenza: I dati sono rappresentati uniformemente nella base di dati. I valori TRUE/FALSE - 0/1 - SI/NO sono stati inseriti utilizzando la sintassi TRUE/FALSE in tutti i database.

Attualità dei dati: I dati non sono aggiornati al 2013 (anno attuale), ma risalgono tutti ad anni passati. Nel caso di IMDB i dati sono aggiornati al 2009, nel caso di UWCSE i dati sono aggiornati al 2004 e nel caso di WEBKB
i dati sono aggiornati al 1997. Ad ogni modo ciò non rappresenta un problema visto che l’obiettivo del KDD process è quello di valutare due versioni di un algoritmo di Data Mining.

Bewertung der Ergebnisse
------------------------

Der Jury ist bekannt, welche von den 11.177 zu klassifizierenden
E-Mails tatsächlich Nicht-Spam-Mail oder Spam-Mail ist. Genauer
gesagt, stammen alle Daten aus einer Stichprobe von insgesamt
19.177 E-Mails.

Die eingesandten Ergebnisse werden mit der bekannten Information über
die tatsächliche Zuordnung der E-Mails verglichen und der Anteil der
nicht ausgefilterten Spam-Mails bestimmt. Gleichzeitig wird der Anteil
der ausgefilterten Nicht-Spam-Mails ermittelt und die Einhaltung
der 1 \% Klausel (siehe oben) überprüft. Sieger ist der Teilnehmer
oder die Teilnehmerin, welche(r) unter Einhaltung der 1 \% Klausel die
wenigsten Spam-Mails zustellt. Teilnehmer, die die 1 \% Klausel ver-
letzen, werden nicht gewertet.

La giuria non è noto quali classificare le e-mail è in realtà la posta o spam 11.177 non-spam. In particolare, tutti i dati provengono da un campione totale di 19.177 messaggi di posta elettronica.

I risultati presentati sono confrontati con le informazioni note circa l'assegnazione effettiva delle e-mail e determinata la percentuale di spam non filtrati. Allo stesso tempo, la percentuale di filtrato non spam viene rilevato e il rispetto della clausola 1 \% (vedi sopra) selezionata. Il vincitore è il partecipante o la partecipante, che (r) manda le mail di spam in meno rispetto alla clausola 1 \%. I partecipanti, la clausola 1 \% ferire non saranno conteggiati.


I dati contenuti nei tre database sono quasi sempre di alta qualità, ma è bene soffermarsi sulle caratteristiche che questi devono avere:
\begin{itemize}
	\item a;
\end{itemize}

Accuratezza: I valori contenuti nei tre database non sono cambiati nel tempo a parte per quanto riguarda il database WEBKB che potrebbe avere subito un aggiornamento nel tempo, trattandosi di pagine web. Ad ogni modo ciò non rappresenta un problema visto che l'obiettivo del KDD process è quello di valutare due versioni di un algoritmo di Data Mining.
Completezza: Non sempre tutti i dati sono disponibili. Ci sono alcuni missing values nei tre database.
IMDB:nella relazione "movies" assume valore nullo l'attributo "budget" (l'informazione non è disponibile).
UWCSE: nella relazione "persons" spesso assume valore nullo l'attributo "phase" (quando la persona è una matricola o un professore) e nella relazione "prof" talvolta assume valore nullo l'attributo "position" (non è specificata la posizione del professore). 
Consistenza: I dati sono rappresentati uniformemente nella base di dati. I valori TRUE/FALSE - 0/1 - SI/NO sono stati inseriti utilizzando la sintassi TRUE/FALSE in tutti i database. 
Attualità dei dati: I dati non sono aggiornati al 2013 (anno attuale), ma risalgono tutti ad anni passati. 
Nel caso di IMDB i dati sono aggiornati al 2009, nel caso di UWCSE i dati sono aggiornati al 2004 e nel caso di WEBKB i dati sono aggiornati al 1997. Ad ogni modo ciò non rappresenta un problema visto che l’obiettivo del KDD process è quello di valutare due versioni di un algoritmo di Data Mining.Non sono richiesti dati esterni da integrare nelle tre basi di dati.

\section{Esportazione dei dati}
Per ognuno dei tre database saranno mostrati di seguito alcuni grafici che mostrano l’andamento di alcuni attributi.


